\RequirePackage[l2tabu, orthodox]{nag}
\documentclass[12pt]{article}
\usepackage[utf8]{inputenc}
\usepackage[T1]{fontenc}

\pagenumbering{gobble}

\usepackage{amsmath, amsfonts, amssymb}
\usepackage{textcomp}

\begin{document}

\begin{center}
   \Large\textbf{Pseudorandom Repetition}
\end{center}

I've been working on a program that needs lots of random numbers, so I wrote a Linear Congruential Generator to make them. A LCG is a kind of pseudorandom number generator where the next value \(n_{i+1}\) equals a times the previous n, \(n_i\), plus \(c\), mod \(m\).
\[n_{i+1} = a n_i + c~(\bmod~m)\]
I picked a LCG because it was simple, but LCGs eventually repeat themselves. Given the values \(a\), \(c\), \(m\), and the initial seed \(n_i\), where all four numbers are \(>0\) and \(<2^{63}-1\), compute the period of the generator, as well as the first value to repeat.

\vspace*{.3in} \noindent {\large \bfseries Input Format}\\

The input will consist of four integers, \(a\), \(c\), \(m\), and \(n_i\), representing the scalar, constant, modulus, and initial seed, respectively.

\vspace*{.3in} \noindent {\large \bfseries Output Format}\\

Your output be the period, \(p\) of the given parameters as well as \(n_{i+x}\), the first number to be repeated. Expect the inputs as well as the outputs to be quite large.

\vspace*{.3in} \noindent {\large \bfseries Sample Input}
\begin{verbatim}
97
1
109
5
\end{verbatim}

\vspace*{.3in} \noindent {\large \bfseries Sample Output}
\begin{verbatim}
28
5
\end{verbatim}

\end{document}
